\documentclass{beamer}

% --- PAQUETS ET CONFIGURATION ---
\usepackage[utf8]{inputenc}
\usepackage[T1]{fontenc}
\usepackage[french]{babel}
\usepackage{graphicx}
\usepackage{listings} % Pour le code SQL
\usepackage{xcolor}

% --- THÈME ---
\usetheme{Madrid}
\usecolortheme{beaver} % Thème de couleur (Rouge/Gris)

% --- CONFIGURATION DU CODE SQL ---
\definecolor{codegreen}{rgb}{0,0.6,0}
\definecolor{codegray}{rgb}{0.5,0.5,0.5}
\definecolor{codepurple}{rgb}{0.58,0,0.82}
\definecolor{backcolour}{rgb}{0.95,0.95,0.92}

\lstdefinestyle{mystyle}{
    backgroundcolor=\color{backcolour},   
    commentstyle=\color{codegreen},
    keywordstyle=\color{blue},
    numberstyle=\tiny\color{codegray},
    stringstyle=\color{codepurple},
    basicstyle=\ttfamily\footnotesize,
    breakatwhitespace=false,         
    breaklines=true,                 
    captionpos=b,                    
    keepspaces=true,                 
    numbers=left,                    
    numbersep=5pt,                  
    showspaces=false,                
    showstringspaces=false,
    showtabs=false,                  
    tabsize=2,
    language=SQL
}
\lstset{style=mystyle}

% --- INFOS DE LA PRÉSENTATION ---
\title[Projet Base de Données]{Gestion de Commandes et Fidélité}
\subtitle{Application "UberMiam"}

\author[Abeille, Guiot, Picarel, Bely]{ 
    Thibault ABEILLE \and
    Numa GUIOT \and
    Enzo PICAREL \and
    Raphaël BELY
}
\date{\today}

\begin{document}

% ------------------------------------------------
% SLIDE DE TITRE
% ------------------------------------------------
\begin{frame}
    \titlepage
\end{frame}

% ------------------------------------------------
% SLIDE 1 : MODÈLE CONCEPTUEL
% ------------------------------------------------
\begin{frame}{1. Modèle Conceptuel (Entité-Association)}
    \begin{columns}
        \column{0.4\textwidth}
        \textbf{Choix de conception :}
        \begin{itemize}
            \item \textbf{Clients / Restaurants} : Entités centrales.
            \item \textbf{Fidélité} : Association porteuse de données (points) entre un Client et un Restaurant.
            \item \textbf{Remises} : Spécialisation (Héritage) en \textit{ItemOffert} ou \textit{Pourcentage}.
            \item \textbf{Formules} : Composition complexe d'items.
        \end{itemize}

        \column{0.6\textwidth}
        \begin{figure}
            \centering
            % REMPLACEZ par le nom de votre image (ex: diagramme.png)
            % Si vous n'avez pas l'image, le cadre 'demo' s'affichera
            \includegraphics[width=\textwidth, height=0.6\textheight, keepaspectratio]{schema_entite_association.jpg}
            \caption{Diagramme E/A}
        \end{figure}
    \end{columns}
\end{frame}

% ------------------------------------------------
% SLIDE 2 : MODÈLE RELATIONNEL
% ------------------------------------------------
\begin{frame}{2. Modèle Relationnel (Normalisé)}
    passage au schéma relationnel en respectant la 3ème Forme Normale (3FN).
    
   \begin {figure}
        \centering
        % REMPLACEZ par le nom de votre image (ex: relationnel.png)
        % Si vous n'avez pas l'image, le cadre 'demo' s'affichera
        \includegraphics[width=0.9\textwidth, height=0.6\textheight, keepaspectratio]{schéma_relationnel.jpg}
        \caption{Schéma Relationnel}
    \end{figure}
    
\end{frame}

% ------------------------------------------------
% SLIDE 3 : REQUÊTE COMPLEXE
% ------------------------------------------------
% ------------------------------------------------
% SLIDE 3 : REQUÊTE COMPLEXE (Disponibilité Formules)
% ------------------------------------------------
\begin{frame}[fragile]{3. Exemple de Requête Complexe}
    \textbf{Objectif :} Afficher les formules disponibles en temps réel (selon le jour et l'heure).
    
    \begin{lstlisting}[basicstyle=\tiny\ttfamily]
SELECT 
    f.formule_id, f.nom, f.prix, ci.nom AS nom_categorie
FROM formules AS f
INNER JOIN composer_formules AS cf ON cf.formule_id = f.formule_id
INNER JOIN categories_items AS ci 
    ON ci.categorie_item_id = cf.categorie_item_id
WHERE f.restaurant_id = ?  
  AND (
    -- 1. Aucune restriction (formule toujours dispo)
    NOT EXISTS (
        SELECT 1 FROM avoir_conditions_formules 
        WHERE formule_id = f.formule_id
    )
    OR 
    -- 2. Restrictions respectees (jour et heure actuels)
    EXISTS (
       SELECT 1
       FROM avoir_conditions_formules AS acf
       JOIN conditions_formules AS cond 
         ON acf.condition_formule_id = cond.condition_formule_id
       WHERE acf.formule_id = f.formule_id 
         AND cond.jour_disponibilite = EXTRACT(ISODOW FROM CURRENT_DATE)
         AND CURRENT_TIME BETWEEN cond.creneau_horaire_debut 
                              AND cond.creneau_horaire_fin
   ))
ORDER BY f.prix
    \end{lstlisting}
    
    \vspace{0.1cm}
    \tiny 
    \textit{Note : Utilisation de \texttt{NOT EXISTS} / \texttt{EXISTS} et des fonctions temporelles \texttt{EXTRACT(ISODOW)} et \texttt{CURRENT\_TIME} pour le filtrage dynamique.}
\end{frame}
% ------------------------------------------------
% SLIDE 4 : L'APPLICATION
% ------------------------------------------------
\begin{frame}{4. Démonstration de l'Application}
    \begin{columns}
        \column{0.5\textwidth}
        \textbf{Architecture Technique :}
        \begin{itemize}
            \item \textbf{Backend} : PHP (Sans framework, modèle MVC artisanal).
            \item \textbf{BDD} : PostgreSQL + PostGIS (Coordonnées GPS).
            \item \textbf{Frontend} : HTML5 / CSS3.
        \end{itemize}
        
        \vspace{0.5cm}
        
        \textbf{Fonctionnalités clés :}
        \begin{itemize}
            \item Panier dynamique.
            \item Calcul automatique des points.
            \item Mode "Invité" vs "Client".
        \end{itemize}

        \column{0.5\textwidth}
        % \begin{figure}
        %     \centering
        %     % REMPLACEZ par une capture d'écran de votre site
        %     \includegraphics[width=\textwidth, angle=-5]{} 
        %     \caption{Interface de commande}
        % \end{figure}
    \end{columns}
\end{frame}

\end{document}
